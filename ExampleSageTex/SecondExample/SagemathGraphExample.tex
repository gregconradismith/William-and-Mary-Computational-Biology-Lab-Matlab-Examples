\documentclass{article}
\usepackage{sagetex}
\usepackage{tikz,tkz-graph,tkz-berge}
\thispagestyle{empty}
\begin{document}

\begin{sagesilent}
k = 2
G=Graph({0: {1: '01', 2: '02', 3: '03'}, 1: {4: '14', 5: '15'}, 2: {6: '26', 4: '24'}, 3: {5: '35', 6: '36'}, 4: {7: '47'}, 5: {7: '57'}, 6: {7: '67'} }); graph_name='Q3' # Cube with natural notation 
v = G.order() 
e = G.size()
show(G,figsize=[3,3])
G.show(title=graph_name,graph_border=True,edge_labels=True,figsize=[3,3])
b = e-v+1
print 'k =', k
print graph_name, 'has e =',e,', v =', v, 'and b =', b 


E = e*binomial(k+v-2,k-1) # note v-1=e since beta=0 for T
V = binomial(k+v-1,k)
beta = E-V+1;
R = PolynomialRing(ZZ,v,'a') # Ring of vertices of G^(k) as monomials with indeterminants associated to vertices of T/G (note V[T]=V(G))
R.inject_variables(verbose=false)
av = sum([ R.gen(i) for i in range(v) ])
f = av^k; f1 = av^(k-1); # f2 = av^(k-2); 
F = f.monomials(); F1 = f1.monomials(); # F2 = f2.monomials()
#shortprint(F,'F ='); shortprint(F1,'F1 ='); # shortprint(F2,'F2 =')
dF = dict((F[i],i) for i in range(len(F))) # dictionary of monomials in Mk 
dF1 = dict((F1[i],i) for i in range(len(F1))) # dictionary of monomials in Mk-1 
Gk=Graph(vertex_labels=true)
for f in F:
    Gk.add_vertex(name=f) 
    #print G.edges()
    print
    for edge in G.edges(): # every edge of G
        for jmk1 in F1: # every F1
            vm,vp,lab = edge
            x = jmk1*R.gen(vm)
            y = jmk1*R.gen(vp)
            Gk.add_edge(x,y,edge[2])
H=Gk
H.set_latex_options(scale=4,graphic_size=(3.1,3.1))
\end{sagesilent}

\begin{center}
\begin{tikzpicture}
\GraphInit[vstyle=Normal]
\SetVertexNormal[Shape=rectangle,LineWidth = 1pt]
\tikzset{EdgeStyle/.append style = {opacity=0.6,color = blue!60, line width=2pt}}
\sage{H}
\end{tikzpicture}
\end{center}

