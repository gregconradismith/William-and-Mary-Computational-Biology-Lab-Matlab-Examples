
\exercise[(Solving a difference equation)]{6.4}

Consider the difference equation $U^{n+2} = U^n$ with starting values $U^0$
and $U^1$.  The solution is clearly
\[
U^n = \begin{choices} U^0  & \text{if $n$ is even},\\
               U^1  & \text{if $n$ is odd}.
\end{choices}
\]
Using the roots of the characteristic polynomial and the approach of Section
6.4.1, another representation of this solution can be found:
\[
U^n = (U^0 + U^1) + (U^0 - U^1)(-1)^n.
\]
Now consider the difference equation $U^{n+4}=U^n$ with four starting values
$U^0,~U^1,~U^2,~U^3$.  Use the roots of the characteristic polynomial to
find an analogous represenation of the solution to this equation.
