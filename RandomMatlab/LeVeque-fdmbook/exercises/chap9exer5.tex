
\exercise[(Diffusion and decay)]{9.5}

Consider the PDE
\eqlex{diffdecay}
u_t = \kappa u_{xx} - \gamma u,
\end{equation}
which models a diffusion with decay provided $\kappa>0$ and $\gamma>0$.  
Consider methods of the form
\eqlex{ddtheta}
U_j^{n+1} =U_j^n+{k\over 2h^2} [U_{j-1}^n-2U_j^n +U_{j-1}^n
+U_j^{n+1} - 2U_j^{n+1} + U_j^{n+1}]
- k\gamma[(1-\theta)U_j^n + \theta U_j^{n+1}]
\end{equation}
where $\theta$ is a parameter.  In particular, if $\theta=1/2$ then the
decay term is modeled with the same centered-in-time approach as the
diffusion term and the method can be obtained by applying the Trapezoidal
method to the MOL formulation of the PDE.   If $\theta=0$ then the decay
term is handled explicitly.  For more general reaction-diffusion equations
it may be advantageous to handle the reaction terms explicitly since these
terms are generally nonlinear, so making them implicit would require solving
nonlinear systems in each time step (whereas handling the diffusion term
implicitly only gives a linear system to solve in each time step).

\begin{enumerate}
\item By computing the local truncation error, show that this method is
$\bigo(k^p+h^2)$ accurate, where $p=2$ if $\theta = 1/2$ and $p=1$
otherwise.
\item Using von Neumann analysis, show that this method is unconditionally
stable if $\theta \geq 1/2$.
\item Show that if $\theta = 0$ then the method is stable provided $k\leq
2/\gamma$, independent of $h$.
\end{enumerate} 

