% Template file for an a0 landscape poster.
% Written by Graeme, 2001-03 based on Norman's original microlensing
% poster.
%
% See discussion and documentation at
% <http://www.astro.gla.ac.uk/users/norman/docs/posters/> 
%
% $Id: poster-template-landscape.tex,v 1.2 2002/12/03 11:25:46 norman Exp $


% Default mode is landscape, which is what we want, however dvips and
% a0poster do not quite do the right thing, so we end up with text in
% landscape style (wide and short) down a portrait page (narrow and
% long). Printing this onto the a0 printer chops the right hand edge.
% However, 'psnup' can save the day, reorienting the text so that the
% poster prints lengthways down an a0 portrait bounding box.
%
% 'psnup -w85cm -h119cm -f poster_from_dvips.ps poster_in_landscape.ps'

\documentclass[a0]{a0poster}
% You might find the 'draft' option to a0 poster useful if you have
% lots of graphics, because they can take some time to process and
% display. (\documentclass[a0,draft]{a0poster})

\pagestyle{empty}
\setcounter{secnumdepth}{0}

% The textpos package is necessary to position textblocks at arbitary 
% places on the page.
\usepackage[absolute]{textpos}

% Graphics to include graphics. Times is nice on posters, but you
% might want to switch it off and go for CMR fonts.
% Greg commented this out 
% \usepackage{graphics,wrapfig,times}

% Greg's additions 
%%
%% MATH DEFINITIONS   
%%

\def\({\left(}
\def\){\right)}

\def\inv#1{{\frac{1}{#1}}}
\def\ddt#1{\der{#1}{t}}
\def\ddx#1{\der{#1}{x}}
\def\ddz#1{\der{#1}{z}}
\def\der#1#2{{\frac{d#1}{d#2}}}
\def\derder#1#2{{d^2\frac{#1}{d{#2}^2}}}
\def\norm#1{\labs #1 \rabs}
\def\pder#1#2{{\frac{\partial#1}{\partial#2}}}
\def\pderder#1#2{{\frac{\partial^2#1}{\partial{#2}^2}}}
\def\pdt#1{\pder{#1}{t}}
\def\pdx#1{\pder{#1}{x}}
\def\pdxx#1{\pderder{#1}{x}}
\def\pdz#1{\pder{#1}{z}}
\def\pdzz#1{\pderder{#1}{z}}
\def\ssize#1{{\scriptsize #1}}

\def\bmat{\left[ \begin{array}}
\def\bna{\begin{array}}
\def\bneas{\begin{eqnarray*}}
\def\bnea{\begin{eqnarray}}
\def\bnes{\begin{displaymath}}
\def\bne{\begin{equation}}

\def\emat{\end{array} \right]}
\def\ena{\end{array}}
\def\eneas{\end{eqnarray*}}
\def\enea{\end{eqnarray}}
\def\enes{\end{displaymath}}
\def\ene{\end{equation}} 

\def\half{\frac{1}{2}} 

\def\revrxn#1#2{
\begin{array}{c} 
{#1}\\
\rightleftharpoons \\
{#2}
\end{array}
}

\def\Bold#1{\vspace{0.1in} \noindent {\bf \boldmath #1}}

\def\Sec#1{Section~\ref{#1}}
\def\Secs#1#2{Sections~\ref{#1}--\ref{#2}}
\def\SecsAnd#1#2{Sections~\ref{#1} and \ref{#2}}

\def\Eqn#1{Eq.~\ref{#1}}
\def\Eqns#1#2{Eqs.~\ref{#1}--\ref{#2}}
\def\EqnsAnd#1#2{Eqs.~\ref{#1} and \ref{#2}}

\def\Eq#1{Eq.~\ref{#1}}
\def\Eqs#1#2{Eqs.~\ref{#1}--\ref{#2}}
\def\EqsAnd#1#2{Eqs.~\ref{#1} and \ref{#2}}

\def\Fig#1{Fig.~\ref{#1}}
\def\Figs#1#2{Figs.~\ref{#1}--\ref{#2}}
\def\FigsAnd#1#2{Figs.~\ref{#1} and \ref{#2}}

\def\Figure#1{Figure~\ref{#1}}
\def\Figures#1#2{Figures~\ref{#1}--\ref{#2}}
\def\FiguresAnd#1#2{Figures~\ref{#1} and \ref{#2}}

\def\avg#1{\langle{#1}\rangle}
\def\Parens#1{\left({#1}\right)}

\def\Bold#1{{\bf #1}}
\def\Block#1#2{{\bf #1:} {#2}\\ }

\def\bfcite#1{{\bf\cite{#1}}}

\def\bpi{\boldsymbol \pi}
\def\E{\mathsf E}
\def\P{\mathsf P}
\def\Pr{\mathsf P}
\def\Var{\mathsf{Var}}


%%
%% CALCIUM DEFINITIONS   
%%

% My Definitions

\def\FcepsilonR1{Fc$\epsilon$R1}

\def\Ps{$\ps$}
\def\ps{{\rm s}^{-1}}

\def\Pms{$\pms$}
\def\pms{{\rm ms}^{-1}}

\def\UM{$\uM$}
\def\uM{\mu{\rm M}}

\def\Um{$\um$}
\def\um{\mu{\rm m}}

\def\Us{$\us$}
\def\us{\mu{\rm s}}

\def\Ums{$\ums$}
\def\ums{\mu{\rm m}^2}

\def\Nm{$\nm$}
\def\nm{\mu{\rm m}}

\def\Pumps{$\pumps$}
\def\pumps{\mu{\rm M}^{-1} {\rm s}^{-1}}

\def\Pumpms{$\pumpms$}
\def\pumpms{\mu{\rm M}^{-1} {\rm ms}^{-1}}

\def\UMps{$\uMps$}
\def\uMps{\mu{\rm M}/{\rm s}}

\def\Umpss{$\umpss$}
\def\umpss{\mu{\rm m}/{\rm s}^2}

\def\Umsps{$\umsps$}
\def\umsps{\mu{\rm m}^2/{\rm s}}

\def\Umspms{$\umspms$}
\def\umspms{\mu{\rm m}^2/{\rm ms}}

\def\Ryr{RyR}
\def\ip{{[{\rm IP}_3]}}

\def\Ip{${\rm IP}_3$}
\def\Ipr{${\rm IP}_3{\rm R}$}

\def\cad{[{\rm Ca}^{\rm 2+}]_{\rm d}}
\def\Cad{$\cad$}

\def\cabulk{[{\rm Ca}^{\rm 2+}]_{\rm bulk}}
\def\Cabulk{$\cabulk$}

\def\cainf{[{\rm Ca}^{\rm 2+}]_{\infty}}
\def\Cainf{$\cainf$}

\def\cai{[{\rm Ca}^{\rm 2+}]_{\rm i}}
\def\Cai{$\cai$}

\def\caext{[{\rm Ca}^{\rm 2+}]_{\rm ext}}
\def\Caext{$\caext$}

\def\cae{[{\rm Ca}^{\rm 2+}]_{\rm ER}}
\def\Cae{$\cae$}

\def\cat{[{\rm Ca}^{\rm 2+}]_{\rm T}}
\def\Cat{$\cat$}

\def\ve{{\varepsilon}}
\def\Bj{${\rm B}_j$}
\def\B{${\rm B}$}
\def\Bt{$\bt$}
\def\Ca{${\rm Ca}^{\rm 2+}$}
\def\K{${K}^{+}$}
\def\Cabj{${\rm CaB}_j$}
\def\Cab{${\rm CaB}$}
\def\Kj{$\kj$}
\def\Kjm{$\kjm$}
\def\Kjp{$\kjp$}
\def\Kmm{$\kmm$}
\def\Kkm{$\kkm$}
\def\Kmp{$\kmp$}
\def\Km{$\km$}
\def\Kp{$\kp$}
\def\bjt{\bj_T}
\def\b{[{\rm B}]}
\def\bj{[{\rm B}_j]}
\def\bmt{\bm_T}
\def\bt{\b_{\rm T}}
\def\bm{[B_m]}
\def\cabj{[{\rm CaB}_j]}
\def\cab{[{\rm CaB}]}
\def\ca{[{\rm Ca}^{2+}]}
\def\dc{D_c}
\def\dj{D_j}
\def\dm{D_m}
\def\db{D_b}
\def\kj{K_j}
\def\kjm{k^-_j}
\def\kjp{k^+_j}
\def\kmm{k^-_m}
\def\kmp{k^+_m}
\def\kkm{K_m}
\def\km{k^-}
\def\kp{k^+}

\def\wm{w^-}
\def\wp{w^+}
\def\minf{m_\infty}

% other ions

\def\mg{[{\rm Mg}^{2+}]}
\def\Mg{${\rm Mg}^{\rm 2+}$}





\def\Bold#1{{\bf #1}}
\def\Block#1#2{{\bf #1:} {#2}\\ }

\def\bfcite#1{{\bf\cite{#1}}}

\usepackage{epsfig}
\usepackage{amsmath}
\usepackage{amssymb}

% These colours are tried and tested for titles and headers. Don't
% over use color!
\usepackage{color}
\definecolor{DarkBlue}{rgb}{0.1,0.1,0.5}
\definecolor{Red}{rgb}{0.9,0.0,0.1}

% see documentation for a0poster class for the size options here
\let\Textsize\normalsize
\def\Head#1{\noindent\hbox to \hsize{\hfil{\LARGE\color{DarkBlue} #1}}\bigskip}
\def\LHead#1{\noindent{\LARGE\color{DarkBlue} #1}\bigskip}
\def\CHead#1{\begin{center} {\LARGE\color{DarkBlue} #1} \end{center} \bigskip}
\def\Subhead#1{\noindent{\large\color{DarkBlue} #1}\bigskip}

%\def\Title#1{\noindent{\VeryHuge\color{Red} #1}}
\def\Title#1{\begin{center} {\sc \VeryHuge\color{Red} #1} \end{center} }


% Set up the grid
%
% Note that [40mm,40mm] is the margin round the edge of the page --
% it is _not_ the grid size. That is always defined as 
% PAGE_WIDTH/HGRID and PAGE_HEIGHT/VGRID. In this case we use
% 23 x 12. This gives us three columns of width 7 boxes, with a gap of
% width 1 in between them. 12 vertical boxes is a good number to work
% with.
%
% Note however that texblocks can be positioned fractionally as well,
% so really any convenient grid size can be used.
%
\TPGrid[40mm,40mm]{23}{12}      % 3 cols of width 7, plus 2 gaps width 1

\parindent=0pt
\parskip=0.5\baselineskip

\begin{document}

% Understanding textblocks is the key to being able to do a poster in
% LaTeX. In
%
%    \begin{textblock}{wid}(x,y)
%    ...
%    \end{textblock}
%
% the first argument gives the block width in units of the grid
% cells specified above in \TPGrid; the second gives the (x,y)
% position on the grid, with the y axis pointing down.

% You will have to do a lot of previewing to get everything in the 
% right place.

% This gives good title positioning for a portrait poster.
% Watch out for hyphenation in titles - LaTeX will do it
% but it looks awful.
\begin{textblock}{23}(0,0)

\Title{ Calcium puff-like dynamics in Markov chain models \\[0.5in] of instantaneously coupled intracellular calcium channels }

\end{textblock}

\begin{textblock}{23}(0,1.0)
\CHead{Gregory D.~Smith \\ 
Department of Applied Science, College of William and Mary, Williamsburg, VA 23187\\
({\tt greg@as.wm.edu})}

\end{textblock}


% Uni logo in the top right corner. A&A in the bottom left. Gives a
% good visual balance, but you may want to change this depending upon
% the graphics that are in your poster.
\begin{textblock}{2}(0,10)
Your logo here
%\includegraphics{/usr/local/share/images/AandA.epsf}
\end{textblock}

\begin{textblock}{2}(21.2,0)
Another logo here
%\resizebox{2\TPHorizModule}{!}{\includegraphics{/usr/local/share/images/GUVIu/GUVIu.eps}}
\end{textblock}


\begin{textblock}{7}(0,2.5)
\hrule\medskip
\CHead{Abstract}

The stochastic activation and inactivation of intracellular calcium (\Ca)
channels gives rise to localized \Ca\ elevations known as \Ca\ `puffs' and
`sparks.' Beginning with minimal stochastic models of \Ca\ regulation of
individual \Ca\ channels, we derive a formalism for simulation and analysis of
the dynamics of \Ca\ regulation at ``\Ca\ release sites,'' collections of 5-50
channels that give rise to puffs and sparks.  The method involves selecting a
continuous time Markov model for an intracellular \Ca\ channel of interest,
choosing a spatial location for each channel (e.g., from a hypothetical
distribution), and calculating the \Ca\ `microdomain' associated with each
possible configuation of the release site (i.e., which channels are open or
closed).  Because stochastic ion channel models have transition probabilities
that depend on the local \Ca\ concentration, this procedure expands the Markov
model for a single channel into a model of the release site as a collective
entity.  For tightly clustered \Ca\ release sites, the \Ca\ microdomain can be
estimated in a computationally efficient manner as the steady-state solution to
a system of reaction-diffusion equations representing the buffered diffusion of
intracellular \Ca\ from a conselation of point sources.  When biophysically
reastic models of intracellular \Ca\ channels---e.g., the DeYoung-Keizer model
of the inositol 1,4,5-trisphosphate receptor (\Ipr)---are used as the starting
point, this method allows the analysis of equilibrium and non-equilibrium
properties of \Ip-sensitive \Ca\ release sites.  We find that the equilibrium
open probability of \Ip-sensitive \Ca\ release sites can depend biphasically on
the effective density of the release site, a dimensionless quantity that
accounts for release site size, the number of channels per release site, and
the length constant associated with \Ca\ buffers.  The distinct \Ca\ regulatory
properties of \Ipr\ subtypes are shown to influence the collective behavior of
release sites composed of a homogenous population of \Ipr\ subtypes.

\bigskip
\hrule
\end{textblock}


\begin{textblock}{7}(8,2.5)
\hrule\medskip
\CHead{ Q-matrix expansion for instantaneosly coupled channels }
 
If we take as given the interaction matrix $C = \( c_{ij} \)$ giving the
increase in local \Ca\ experienced by channel $j$ when channel $i$ is open
and continue to write the background \Ca\ concentration that is experienced by all
channels even when all are closed as $c_\infty$ (kept separate from $C$), then
for two interacting two-state channels, $C$ is $2 \times 2$ and the expanded Q-matrix
becomes \bne Q^{(2)} = \left( \begin{array}{cccc} \diamond & k^+ c_\infty^\eta & k^+
c_\infty^\eta & \cdot \\ k^- & \diamond & \cdot & k^+ \( c_\infty + c_{21} \)^\eta \\ k^-
& \cdot & \diamond & k^+ \( c_\infty + c_{12} \)^\eta \\ \cdot & k^- & k^- &
\diamond \\ \end{array} \right) \label{TWOSTATEQ2}  \ene while for and for three interacting
channels $C$ is $3 \times 3$ and we have \footnotesize \bne Q^{(3)} = \left(
\begin{array}{cccccccc} \diamond & k^+ c_\infty^\eta & k^+ c_\infty^\eta & \cdot & k^+
c_\infty^\eta & \cdot & \cdot & \cdot \\ k^- & \diamond & \cdot & k^+ \( c_\infty +
c_{12} \)^\eta & \cdot & k^+ \( c_\infty + c_{13} \)^\eta & \cdot & \cdot \\ k^- & \cdot
& \diamond & k^+ \( c_\infty + c_{21} \)^\eta & \cdot & \cdot & k^+ \( c_\infty +
c_{23} \)^\eta & \cdot \\ \cdot & k^- & k^- & \diamond & \cdot & \cdot & \cdot & k^+
\( c_\infty + c_{13} + c_{23} \)^\eta \\ k^- & \cdot & \cdot & \cdot & \diamond &
k^+ \( c_\infty + c_{31} \)^\eta & k^+ \( c_\infty + c_{32} \)^\eta & \cdot \\ \cdot &
k^- & \cdot & \cdot & k^- & \diamond & \cdot & k^+ \( c_\infty + c_{12} +
c_{32} \)^\eta \\ \cdot & \cdot & k^- & \cdot & k^- & \cdot & \diamond & k^+ \(
c_\infty + c_{13} + c_{23} \)^\eta \\ \cdot & \cdot & \cdot & k^- & \cdot & k^- &
k^- & \diamond \\ \end{array} \right)
\label{TWOSTATEQ3}
\ene \normalsize
Notice how the \Ca\ microdomain complicates the expanded Q-matrix formalism.

\bigskip
\hrule
\end{textblock}

\begin{textblock}{7}(16,2.5)
\hrule\medskip
\CHead{ \Ca\ release site simulations using 19 four-state models } 

\begin{center}
\includegraphics{FigSSRS27.eps}
\end{center}

\Ca\ release site simulations using 19 four-state models
(\Eq.~\ref{QFOURSTATE}) show repetitive cycles of activation and inactivation
reminicent of \Ca\ puffs.  Total time:  2 s ({\it left}) and 40 ms ({\it
right}).  Parameters used: $R$ = 0.5 \Um, $r_d$ = 0.05 \Um, $\eta$ = 2,
$c_\infty$ = 50 nM; and $k_i^\pm$ as in \Fig{FigSSRS30}B.

\bigskip
\hrule
\end{textblock}

\begin{textblock}{15}(0,7.5)
\hrule\medskip

\CHead{ Memory-efficient Kronecker operations}
 
The stack operator maps an $n \times m$ matrix into an $nm \times 1$ vector.
That is, if $X$ is a $n \times m$ matrix comprising $m$ column vectors (${\bf
x}_1, {\bf x}_2, \cdots , {\bf x}_m$), where each ${\bf x}_i$ is an $n \times
1$ vector, \[ X = \( \begin{array}{cccc} {\bf x}_1 & {\bf x}_2 & \cdots & {\bf
x}_m \\ \end{array} \)_{n \times m} \] then the stack operator applied to $X$
gives \[ X^S = {\bf x} = \( \begin{array}{c} {\bf x}_1 \\ {\bf x}_2 \\ \vdots
\\ {\bf x}_m \\ \end{array} \) \] A matrix-vector product of the form $\( A
\otimes B \) {\bf x}$ can be efficiently calculated using the stack operation,
it's inverse, and the identity, \bne \( A \otimes B \) {\bf x} = \( A \otimes B
\) X^S = \( B X A^T \)^S \label{FASTKRONSUM}\ene Reshaping $1 \times m_a m_b$
vector ${\bf x}$ to form a $m_b \times m_a$ matrix $X$ and stacking the result
of two matrix multiplications allows one to avoid peforming the Kronecker
product and constructing the $n_a n_b \times m_a m_b$ matrix $A \otimes B$ and
significantly reduces the number of multiplications involved.  A matrix-vector
product involving a Kronecker sum can be similarly calculated, \[ \( A \oplus B
\) {\bf x}  = \( A \otimes I_B + I_A \otimes B \) X^S = \( I_B X A^T  + B X
I_A^T \)^S = \( X A^T  + B X \)^S \] And, finally, a matrix-vector product
involving $N-1$ Kronecker sums of the $n_a \times n_a$ matrices $A_1, A_2,
\dots , A_N$, that is, \[ \( \bigoplus_{i=1}^N A_i \) {\bf x} = \( \sum_{i=1}^N
I_{A_1} \otimes \cdots \otimes A_i \otimes \cdots \otimes I_{A_N} \) {\bf x} \]
can be calculated.... 

\bigskip
\hrule
\end{textblock}

\begin{textblock}{7}(16,8)

\hrule\medskip
\CHead{Example MATLAB Code}
\hrule\medskip

\begin{verbatim}

function [ y ] = mvkronsum(A,B,x)
% MVKRONSUM calculates the matrix-vector product Y = (KRON(A,IB)+KRON(IA,B))*X
% without constructing KRON(A,IB) or KRON(IA,B).  In the above expression, IA
% and IB are identity matrices of the size of A and B, respectively.
 
[ ma, na ] = size(A);
[ mb, nb ] = size(B);
[ mx, nx ] = size(x);
 
y = zeros(mx,nx);
for i = 1:nx
   X = reshape(x(:,i),[nb ma]);
   y(:,i) = reshape(X*A'+B*X,[mx 1]);
end
\end{verbatim}

\end{textblock}

\end{document}
