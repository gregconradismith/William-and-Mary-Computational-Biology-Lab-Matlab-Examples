% Template file for an a0 landscape poster.
% Written by Graeme, 2001-03 based on Norman's original microlensing
% poster.
%
% See discussion and documentation at
% <http://www.astro.gla.ac.uk/users/norman/docs/posters/> 
%
% $Id: poster-template-landscape.tex,v 1.2 2002/12/03 11:25:46 norman Exp $


% Default mode is landscape, which is what we want, however dvips and
% a0poster do not quite do the right thing, so we end up with text in
% landscape style (wide and short) down a portrait page (narrow and
% long). Printing this onto the a0 printer chops the right hand edge.
% However, 'psnup' can save the day, reorienting the text so that the
% poster prints lengthways down an a0 portrait bounding box.
%
% 'psnup -w85cm -h119cm -f poster_from_dvips.ps poster_in_landscape.ps'

\documentclass[a0]{a0poster}
% You might find the 'draft' option to a0 poster useful if you have
% lots of graphics, because they can take some time to process and
% display. (\documentclass[a0,draft]{a0poster})

\pagestyle{empty}
\setcounter{secnumdepth}{0}

% The textpos package is necessary to position textblocks at arbitary 
% places on the page.
\usepackage[absolute]{textpos}

% Graphics to include graphics. Times is nice on posters, but you
% might want to switch it off and go for CMR fonts.
% \usepackage{graphics,wrapfig,times}

%%
%% MATH DEFINITIONS   
%%

\def\({\left(}
\def\){\right)}

\def\inv#1{{\frac{1}{#1}}}
\def\ddt#1{\der{#1}{t}}
\def\ddx#1{\der{#1}{x}}
\def\ddz#1{\der{#1}{z}}
\def\der#1#2{{\frac{d#1}{d#2}}}
\def\derder#1#2{{d^2\frac{#1}{d{#2}^2}}}
\def\norm#1{\labs #1 \rabs}
\def\pder#1#2{{\frac{\partial#1}{\partial#2}}}
\def\pderder#1#2{{\frac{\partial^2#1}{\partial{#2}^2}}}
\def\pdt#1{\pder{#1}{t}}
\def\pdx#1{\pder{#1}{x}}
\def\pdxx#1{\pderder{#1}{x}}
\def\pdz#1{\pder{#1}{z}}
\def\pdzz#1{\pderder{#1}{z}}
\def\ssize#1{{\scriptsize #1}}

\def\bmat{\left[ \begin{array}}
\def\bna{\begin{array}}
\def\bneas{\begin{eqnarray*}}
\def\bnea{\begin{eqnarray}}
\def\bnes{\begin{displaymath}}
\def\bne{\begin{equation}}

\def\emat{\end{array} \right]}
\def\ena{\end{array}}
\def\eneas{\end{eqnarray*}}
\def\enea{\end{eqnarray}}
\def\enes{\end{displaymath}}
\def\ene{\end{equation}} 

\def\half{\frac{1}{2}} 

\def\revrxn#1#2{
\begin{array}{c} 
{#1}\\
\rightleftharpoons \\
{#2}
\end{array}
}

\def\Bold#1{\vspace{0.1in} \noindent {\bf \boldmath #1}}

\def\Sec#1{Section~\ref{#1}}
\def\Secs#1#2{Sections~\ref{#1}--\ref{#2}}
\def\SecsAnd#1#2{Sections~\ref{#1} and \ref{#2}}

\def\Eqn#1{Eq.~\ref{#1}}
\def\Eqns#1#2{Eqs.~\ref{#1}--\ref{#2}}
\def\EqnsAnd#1#2{Eqs.~\ref{#1} and \ref{#2}}

\def\Eq#1{Eq.~\ref{#1}}
\def\Eqs#1#2{Eqs.~\ref{#1}--\ref{#2}}
\def\EqsAnd#1#2{Eqs.~\ref{#1} and \ref{#2}}

\def\Fig#1{Fig.~\ref{#1}}
\def\Figs#1#2{Figs.~\ref{#1}--\ref{#2}}
\def\FigsAnd#1#2{Figs.~\ref{#1} and \ref{#2}}

\def\Figure#1{Figure~\ref{#1}}
\def\Figures#1#2{Figures~\ref{#1}--\ref{#2}}
\def\FiguresAnd#1#2{Figures~\ref{#1} and \ref{#2}}

\def\avg#1{\langle{#1}\rangle}
\def\Parens#1{\left({#1}\right)}

\def\Bold#1{{\bf #1}}
\def\Block#1#2{{\bf #1:} {#2}\\ }

\def\bfcite#1{{\bf\cite{#1}}}

\def\bpi{\boldsymbol \pi}
\def\E{\mathsf E}
\def\P{\mathsf P}
\def\Pr{\mathsf P}
\def\Var{\mathsf{Var}}


%%
%% CALCIUM DEFINITIONS   
%%

% My Definitions

\def\FcepsilonR1{Fc$\epsilon$R1}

\def\Ps{$\ps$}
\def\ps{{\rm s}^{-1}}

\def\Pms{$\pms$}
\def\pms{{\rm ms}^{-1}}

\def\UM{$\uM$}
\def\uM{\mu{\rm M}}

\def\Um{$\um$}
\def\um{\mu{\rm m}}

\def\Us{$\us$}
\def\us{\mu{\rm s}}

\def\Ums{$\ums$}
\def\ums{\mu{\rm m}^2}

\def\Nm{$\nm$}
\def\nm{\mu{\rm m}}

\def\Pumps{$\pumps$}
\def\pumps{\mu{\rm M}^{-1} {\rm s}^{-1}}

\def\Pumpms{$\pumpms$}
\def\pumpms{\mu{\rm M}^{-1} {\rm ms}^{-1}}

\def\UMps{$\uMps$}
\def\uMps{\mu{\rm M}/{\rm s}}

\def\Umpss{$\umpss$}
\def\umpss{\mu{\rm m}/{\rm s}^2}

\def\Umsps{$\umsps$}
\def\umsps{\mu{\rm m}^2/{\rm s}}

\def\Umspms{$\umspms$}
\def\umspms{\mu{\rm m}^2/{\rm ms}}

\def\Ryr{RyR}
\def\ip{{[{\rm IP}_3]}}

\def\Ip{${\rm IP}_3$}
\def\Ipr{${\rm IP}_3{\rm R}$}

\def\cad{[{\rm Ca}^{\rm 2+}]_{\rm d}}
\def\Cad{$\cad$}

\def\cabulk{[{\rm Ca}^{\rm 2+}]_{\rm bulk}}
\def\Cabulk{$\cabulk$}

\def\cainf{[{\rm Ca}^{\rm 2+}]_{\infty}}
\def\Cainf{$\cainf$}

\def\cai{[{\rm Ca}^{\rm 2+}]_{\rm i}}
\def\Cai{$\cai$}

\def\caext{[{\rm Ca}^{\rm 2+}]_{\rm ext}}
\def\Caext{$\caext$}

\def\cae{[{\rm Ca}^{\rm 2+}]_{\rm ER}}
\def\Cae{$\cae$}

\def\cat{[{\rm Ca}^{\rm 2+}]_{\rm T}}
\def\Cat{$\cat$}

\def\ve{{\varepsilon}}
\def\Bj{${\rm B}_j$}
\def\B{${\rm B}$}
\def\Bt{$\bt$}
\def\Ca{${\rm Ca}^{\rm 2+}$}
\def\K{${K}^{+}$}
\def\Cabj{${\rm CaB}_j$}
\def\Cab{${\rm CaB}$}
\def\Kj{$\kj$}
\def\Kjm{$\kjm$}
\def\Kjp{$\kjp$}
\def\Kmm{$\kmm$}
\def\Kkm{$\kkm$}
\def\Kmp{$\kmp$}
\def\Km{$\km$}
\def\Kp{$\kp$}
\def\bjt{\bj_T}
\def\b{[{\rm B}]}
\def\bj{[{\rm B}_j]}
\def\bmt{\bm_T}
\def\bt{\b_{\rm T}}
\def\bm{[B_m]}
\def\cabj{[{\rm CaB}_j]}
\def\cab{[{\rm CaB}]}
\def\ca{[{\rm Ca}^{2+}]}
\def\dc{D_c}
\def\dj{D_j}
\def\dm{D_m}
\def\db{D_b}
\def\kj{K_j}
\def\kjm{k^-_j}
\def\kjp{k^+_j}
\def\kmm{k^-_m}
\def\kmp{k^+_m}
\def\kkm{K_m}
\def\km{k^-}
\def\kp{k^+}

\def\wm{w^-}
\def\wp{w^+}
\def\minf{m_\infty}

% other ions

\def\mg{[{\rm Mg}^{2+}]}
\def\Mg{${\rm Mg}^{\rm 2+}$}




\usepackage{epsfig}

% These colours are tried and tested for titles and headers. Don't
% over use color!
\usepackage{color}
\definecolor{DarkBlue}{rgb}{0.1,0.1,0.5}
\definecolor{Red}{rgb}{0.9,0.0,0.1}

% see documentation for a0poster class for the size options here
\let\Textsize\normalsize
\def\Head#1{\noindent \hbox to \hsize{\hfil{\LARGE\color{DarkBlue} #1}}\bigskip}
\def\LHead#1{\noindent{\LARGE\color{DarkBlue} #1}\bigskip}
\def\CHead#1{\begin{center} {\LARGE\color{DarkBlue} #1} \end{center} \bigskip}
\def\Subhead#1{\noindent{\large\color{DarkBlue} #1}\bigskip}

%\def\Title#1{\noindent{\VeryHuge\color{Red} #1}}
\def\Title#1{\begin{center} {\VeryHuge\color{Red} #1} \end{center} }


% Set up the grid
%
% Note that [40mm,40mm] is the margin round the edge of the page --
% it is _not_ the grid size. That is always defined as 
% PAGE_WIDTH/HGRID and PAGE_HEIGHT/VGRID. In this case we use
% 23 x 12. This gives us three columns of width 7 boxes, with a gap of
% width 1 in between them. 12 vertical boxes is a good number to work
% with.
%
% Note however that texblocks can be positioned fractionally as well,
% so really any convenient grid size can be used.
%
\TPGrid[40mm,40mm]{23}{12}      % 3 cols of width 7, plus 2 gaps width 1

\parindent=0pt
\parskip=0.5\baselineskip

%-----------------------------------------------------------------%

\begin{document}

\begin{textblock}{23}(0,0)

%\Title{ Modeling of Intracellular Calcium Signaling \\[0.1in] that Accounts for Domain \Ca\ Inactivation }
\Title{ Equilibrium Open Probability Calculations of \\[0.1in] Instantaneously-Coupled Intracellular Calcium Channel Models } 

\end{textblock}

\begin{textblock}{23}(0,1.1)
\CHead{Vien D. Nguyen and Gregory D. Smith\\
Department of Applied Science, College of William and Mary, Williamsburg VA\\
({\tt vdnguy@wm.edu and greg@as.wm.edu})}
\end{textblock}


%-----------------------------------------------------------------%


% Uni logo in the top right corner. A&A in the bottom left. Gives a
% good visual balance, but you may want to change this depending upon
% the graphics that are in your poster.
%\includegraphics{/usr/local/share/images/AandA.epsf}

\begin{textblock}{7}(0,2)
\hrule\medskip
\CHead{Introduction}

The endoplasmic reticulum (ER), a membrane-bound compartment within cells,
is an intracellular \Ca\ store.  
Within the cytosol, a normal [\Ca]\ reaches 0.1 \UM,\ while inside the ER the [\Ca]\ is in the range
of 10--1000 \UM.  
Transient increases of [\Ca]\ in the cytosol (approx.\ 1 \UM) is an important intracellular signal. 
Intracellular \Ca\ channels such as the inositol 1,4,5-trisphosphate receptor (\Ipr) that open and close 
in a \Ca\-regulated fashion are often responsible for these intracellular [\Ca] increases. When these ion channels are 
open or `activated', \Ca\ from inside the ER diffuses through the ER membrane into the cytosol and increases the 
cytosolic [\Ca]. Here I derive the equilibrium open probability for two minimal intracellular \Ca\ channel models.  
In the first model, \Ca\ binds the \Ipr\ and leads to activation; in the second model, a slower process of 
\Ca\ inactivation is also included.  When multiple channels of either type are modeled in a manner that accounts for 
the diffusion of \Ca\ between channels, the equilibrium open probability of this \Ca\ channel cluster or `\Ca\ release site' 
as a collective entity is found to depend on the distance between the channels, the background [\Ca], the `domain' [\Ca] 
when a channel is open, and the details of the single-channel model.


%\begin{center}
%\includegraphics{FIG08_02.eps}
%\\ Reproduced from Jafri and Keizer, 1994.
%\end{center}

\bigskip
\hrule
\end{textblock}

%-----------------------------------------------------------------%


\begin{textblock}{7}(0,5.08)
\CHead{Calculating P$_o^{eq}$ for the Two-State Model}

\begin{minipage}{2in} 
\ \\
\includegraphics{twostate_transition.ps}
\end{minipage}
\begin{minipage}{1in}
\ \\
\end{minipage}
\begin{minipage}{10in} 
The two-state \Ipr\ \Ca\ channel model accounts for \Ca\ activation and includes two rate constants ($k^-$ and $k^+$).
The latter rate constant is bimolecular and the rate of activation is given by $k^+c_\infty$ where $c_\infty$ is 
the background or `bulk' [\Ca]. 
\end{minipage}
\\

Equilibrium open probability for the two-state \Ipr\ model:
\[0 = \frac {dP_o} {dt} = k^+c_\infty P_c - k^-P_o\]
\[0 = \frac {dP_c} {dt} = k^-P_c - k^+c_\infty P_o\]
\[P_c = 1 - P_o \]
\[P_o = \frac {k^+ c_\infty} {k^+(c_\infty + k^-)} = 
\frac {c_\infty} {c_\infty+K} \] 
where $K = k^-/k^+$.

\begin{center}
\begin{minipage}{4in} 
%\includegraphics[scale=0.5]{summerfig1.eps} 
\includegraphics[scale=0.6]{single_twostate.eps} 
\end{minipage}
\begin{minipage}{1in} 
\ \\ 
\end{minipage}
\begin{minipage}{7in} 
As background \Ca\ concentration increases, the closed probability 
goes to zero and the open probability increases. In this simulation, $k^+$ = 0.1 $\uM^{-1}$$ sec^{-1}$ and $k^-$ = 0.1 $sec^{-1}$.

\end{minipage}
\end{center}

\bigskip
\hrule
\end{textblock}

%-----------------------------------------------------------------%

\begin{textblock}{7}(0,10)
\CHead{The Four-State Model Includes Calcium Inactivation}
\begin{center}
\begin{minipage}{2.5in}
\includegraphics{fourstate_transition.ps}
\end{minipage}
\begin{minipage}{1in}
\ \\
\end{minipage}
\begin{minipage}{7.5in}
The four-state \Ipr\ \Ca\ channel model accounts for both \Ca\ activation and inactivation 
and includes eight rate constants ($k^-_i$ and $k^+_i$
where $i \in \{ A, B, C, D \}$. Notice that one of the transitions 
out of the open state involves `domain' \Ca\ ($c_d$). 
\end{minipage}
\end{center}
\medskip\hrule
\end{textblock}

%-----------------------------------------------------------------%

\begin{textblock}{7}(8,2.1)
\hrule\smallskip
\CHead{Calculating P$_o^{eq}$ for the Four-State Model}

Using ODEs similar to the two-state model, the equilibrium open probability is found to be:
\[ P_{o}^{eq} = z_O/D \]
where $D = z_{C1} + z_O + z_{C2} + z_{C3}$ and
\bnea z_{C_1} &=& \kappa_b^+ \kappa_c^- \kappa_d^- + \kappa_a^- \kappa_d^-
\kappa_c^- + \kappa_b^- \kappa_a^- \kappa_d^- + \kappa_c^+ \kappa_b^-
\kappa_a^- \\  z_{O} &=& \kappa_d^+ \kappa_c^+ \kappa_b^- + \kappa_c^-
\kappa_d^- \kappa_a^+ + \kappa_d^- \kappa_a^+ \kappa_b^- + \kappa_a^+
\kappa_b^- \kappa_c^+ \\  z_{C_2} &=& \kappa_b^+ \kappa_d^+ \kappa_c^+ +
\kappa_a^- \kappa_d^+ \kappa_c^+ + \kappa_d^- \kappa_a^+ \kappa_b^+ +
\kappa_a^+ \kappa_b^+ \kappa_c^+ \\ z_{C_3} &=& \kappa_d^+ \kappa_b^+
\kappa_c^- + \kappa_a^- \kappa_d^+ \kappa_c^- + \kappa_b^- \kappa_a^-
\kappa_d^+ + \kappa_a^+ \kappa_b^+ \kappa_c^- \enea
and $\kappa_i^+ = k_i^+c_\infty$, $\kappa_i^- = k_i^-$, and
$\kappa_b^+ = k_b^+c_d$.
\\

\begin{minipage}{3in}
\includegraphics[scale=0.6]{summerfig3_1.eps}
\end{minipage}
\begin{minipage}{2in}
\ \\
\end{minipage}
\begin{minipage}{7in}
As background \Ca\ 
concentration increases, the equilibrium open probability of the four-state model first increases and then decreases.
This bell-shaped curve is typical of the type 1 \Ipr. 
Parameters follow De~Young and Keizer 1992: 
$k_a^+$ = 20 $\uM^{-1}sec^{-1}$, $k_a^-$ = 1.6468 $sec^{-1}$, $k_b^+$ = 0.2 $\uM^{-1} sec^{-1}$, $k_b^-$ = 0.2 $sec^{-1}$, $k_c^+$ = 20 $\uM^{-1}sec^{-1}$, $k_c^-$ = 1.6468 $sec^{-1}$, $k_d^+$ = 0.2 $\uM^{-1}sec^{-1}$.  The eighth rate constant, $k_d^-$, 
is given by a thermodynamic constraint.
\end{minipage}

\bigskip
\hrule
\end{textblock}

%-----------------------------------------------------------------%

\begin{textblock}{7}(8,6.8)
\CHead{Modeling Instantaneously-Coupled Channels}
\begin{itemize}
\item In general, for any channel model, the equilibrium open 
probability can be found by solving: 
\[ 
\pi Q = 0,
\]
\vspace{0.1in} 
where $\pi$\ (a vector with length given by the number of channel states) 
is a `stationary distribution' and $Q$ is the `infinitesimal generator matrix' or  `transition matrix'
for the channel model. 

\item For example, for the two-state \Ca\ channel model, 
\[
Q = \left( \begin{array}{cc}
-k^+ c_\infty & k^+ c_\infty \\
k^- & -k^- \\
\end{array}
\right) 
\]

\item For multiple interacting \Ca\ channels, the equilibrium open probability can also be calculated
by `expanding' the transition matrix, $Q$. 
We assume fast diffusion between channels (i.e., instantaneous coupling), given by 
\[ c_{ij} = \frac {\sigma} {2\pi D r_{ij}} e^{-r_{ij}/\lambda} \]
where 
$c_{ij}$ is the concentration increase above background experienced by channel
$j$ when channel $i$ is open and 
$r_{ij}$ is the distance between the two channels. 
\\

$\sigma$ = source amplitude of \Ca\ channel \\
$r$ = distance between channels \\
$D$ = diffusion constant for \Ca \\ 
$\lambda$ = length constant for \Ca\ domain \\
\\
Note that $c_{ij}$ becomes smaller as $r_{ij}$ becomes larger, 
that is, the interaction between channels is weaker when 
the channels are farther apart.


%where the $\diamond$\ denotes the opposite of the sum of the other entries in each row
%so that $Qe = 0$ where $e$ is a commensurate unit vector. 

\end{itemize}
\bigskip
\hrule
\end{textblock}


%-----------------------------------------------------------------%

\begin{textblock}{7}(16,2)
\hrule\medskip
\Head{Modeling Instantaneously-Coupled Channels (continued)}
\begin{itemize}

\item For multiple interacting \Ca\ channels, the equilibrium open probability can be calculated by `expanding' the transition matrix, $Q$.  For three 
two-state channels, the expanded Q-matrix is  
\\

\footnotesize
\[ 
Q^{(3)} = \left(
\begin{array}{cccccccc} \diamond & k^+ c_\infty & k^+ c_\infty & \cdot & k^+
c_\infty & \cdot & \cdot & \cdot \\ k^- & \diamond & \cdot & k^+ \( c_\infty +
c_{12} \) & \cdot & k^+ \( c_\infty + c_{13} \) & \cdot & \cdot \\ k^- & \cdot
& \diamond & k^+ \( c_\infty + c_{21} \) & \cdot & \cdot & k^+ \( c_\infty +
c_{23} \) & \cdot \\ \cdot & k^- & k^- & \diamond & \cdot & \cdot & \cdot & k^+
\( c_\infty + c_{13} + c_{23} \) \\ k^- & \cdot & \cdot & \cdot & \diamond &
k^+ \( c_\infty + c_{31} \) & k^+ \( c_\infty + c_{32} \) & \cdot \\ \cdot &
k^- & \cdot & \cdot & k^- & \diamond & \cdot & k^+ \( c_\infty + c_{12} +
c_{32} \) \\ \cdot & \cdot & k^- & \cdot & k^- & \cdot & \diamond & k^+ \(
c_\infty + c_{13} + c_{23} \) \\ \cdot & \cdot & \cdot & k^- & \cdot & k^- &
k^- & \diamond \\ \end{array} \right) 
\label{TWOSTATEQ3}
\] 
\normalsize 
\\
\item The expanded Q matrix for three coupled four-state channels can be derived in a similar manner.  here the cluster of channels has a total of $4^3 = 256$ states (so Q$^{(3)}$ has 256x256 elements)!

\end{itemize}

\bigskip
\hrule
\end{textblock}

%-----------------------------------------------------------------%

\begin{textblock}{7}(16,5.5)
\Head{Two-state versus Four-state Multiple Channel Simulation}
\begin{center}
\begin{minipage}{4in}
\includegraphics[scale=0.7]{ptotal_2state_graph.ps}	
\end{minipage}
\begin{minipage}{3in}
\ \\
\end{minipage}
\begin{minipage}{4in}
\includegraphics[scale=0.7]{ptotal_4state_graph_ncoop_1_kconstants.ps}	
%\includegraphics[scale=0.6]{ptotal_4state_graph_DYK_fixedparams.ps}	
\end{minipage}
\end{center}
\ \\ 

The equilibrium open probability of three coupled channels arranged 
in an equilateral triangle as a function of the distance between 
the channels.  The left and right graph are for the two-state and four-state channel models, respectively.  Parameters as in the single channel simulation
above.  As expected, the two-state channels activate each other on average 
as they get close.  The four-state channels 
can either activate or inactivate each other 
on average, depending on their distance. 
However, this inactivation does not appear until the channels' distance 
is smaller than the distance between the pore and binding site of a 
single channel (r$_{ii}$).
\\

\bigskip
\hrule
\end{textblock}

%-----------------------------------------------------------------%
\begin{textblock}{7}(16,9.5)
\CHead{Future Research}


In future work we will explore if it is true in general that closely positioned 
\Ca\ channels are unlikely to inactivate one another, even when the individual 
channel models include \Ca-inactivation.

\bigskip
\hrule
\end{textblock}

%-----------------------------------------------------------------%

\begin{textblock}{7}(16,10.7)
\CHead{Acknowledgments}

This project was supported in part by the College of William and Mary's Physics
REU program, NSF CAREER award \#0133132, and the Thomas F. and Kate Miller
Jeffress Memorial Trust and performed in part using computational
facilities at the College of William and Mary enabled by grants from the NSF
and Sun Microsystems.  Thanks to Chi-Kwong Li, Roy Mathias, and Thomas
Milligan for helpful conversations.  Special thanks goes to Dr. Smith for
getting this project started and helping out on this project.

\bigskip
\hrule
\end{textblock}

\end{document}
