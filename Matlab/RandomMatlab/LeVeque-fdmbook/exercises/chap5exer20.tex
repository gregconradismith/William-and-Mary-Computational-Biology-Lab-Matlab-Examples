
\exercise[($R(z)$ for Runge-Kutta methods)]{5.20}

Consider a general $r$-stage Runge-Kutta method with tableau defined by an
$r\times r$ matrix $A$ and a row vector $b^T$ of length $r$.  Let
\[
Y = \bcm Y_1 \\ Y_2\\ \vdots\\ Y_r\ecm, \qquad
e = \bcm 1\\ 1\\ \vdots\\ 1\ecm
\]
and $z=\lambda k$.  
\begin{enumerate}
\item Show that if the Runge-Kutta method is applied to the equation
$u'=\lambda u$ the formulas (5.34) can be written concisely as
\begin{equation*}
\begin{split}
Y &= U^ne + zAY,\\
U^{n+1} &= U^n + zb^TY,
\end{split}
\end{equation*}
and hence
\eqlex{a}
\qquad U^{n+1} = \left( I + zb^T(I-zA)^{-1}e\right) U^n.
\end{equation}

\item
Recall that by Cramer's rule that if $B$ is an $r\times r$ matrix then the
$i$th element of the vector $y = B^{-1}e$ is given by 
\[
y_i = \frac{\det(B_i)}{\det(B)},
\]
where $B_i$ is the matrix $B$ with the $i$th column replaced by $e$, and
$\det$ denotes the determinant.

In the expression \eqnex{a}, $B=I-zA$ and each element of $B$ is linear in $z$.
From the definition of the determinant it follows that $\det(B)$ will be a
polynomial of degree at most $r$, while $\det(B_i)$ will be a polynomial
of degree at most $r-1$ (since the column vector $e$ does does not involve $z$).

From these facts, conclude that \eqnex{a} 
yields $U^{n+1} = R(z)U^n$ where $R(z)$
is a rational function of degree at most $(r,r)$.

\item Explain why an explicit Runge-Kutta method (for which $A$ is strictly
lower triangular) results in $R(z)$ being a polynomial of degree at most $r$
(i.e., a rational function of degree at most $(r,0)$).

\item Use \eqnex{a} to determine the function $R(z)$ for the TR-BDF2 method
(5.36).  Note that in this case $I-zA$ is lower triangular and you can
compute $(I-zA)^{-1}e$ by forward substitution.  You should get the same
result as in Exercise 5.18(c).
\end{enumerate}


