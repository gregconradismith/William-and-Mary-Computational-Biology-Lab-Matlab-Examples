
\exercise[(accuracy on nonuniform grids)]{2.5}

In Example 1.4 a 3-point approximation to $u''(x_i)$ is determined based on
$u(x_{i-1}),~ u(x_i)$, and $u(x_{i+1})$ (by translating from $x_1$, $x_2$,
$x_3$ to general $x_{i-1}$, $x_i$, and $x_{i+1}$).  
It is also determined that the truncation
error of this approximation is $\frac 1 3 (h_{i-1}-h_i) u'''(x_i) + O(h^2)$,
where $h_{i-1} = x_i-x_{i-1}$ and $h_i = x_{i+1}-x_i$, so the approximation is
only first order accurate in $h$ if $h_{i-1}$ and $h_i$ are  $O(h)$ but
$h_{i-1}\neq h_i$.  

The program {\tt bvp2.m} is based on using 
this approximation at each grid point, as described in Example 2.3.  
Hence on a nonuniform grid the local
truncation error is $O(h)$ at each point, where $h$ is some measure of the
grid spacing (e.g., the average spacing on the grid).  If we
assume the method is stable, then we expect the global error to be $O(h)$ as
well as we refine the grid.

\begin{enumerate}
\item
However, if you run {\tt bvp2.m} you should observe second-order
accuracy, at least provided you take a smoothly varying grid (e.g., set
{\tt gridchoice = 'rtlayer'} in {\tt bvp2.m}).  Verify this.

\item
Suppose that the grid is defined by $x_i = X(z_i)$ where $z_i = ih$ for
$i=0,~1,~\ldots,~m+1$ with $h=1/(m+1)$
is a uniform grid and $X(z)$ is some smooth mapping
of the interval $[0,1]$ to the interval $[a,b]$.  Show that if $X(z)$ is
smooth enough, then the local truncation error is in fact $O(h^2)$.
Hint: $x_i-x_{i-1} \approx hX'(x_i)$.

\item
What average order of accuracy is observed on a random grid?  To test this,
set {\tt gridchoice = 'random'} in {\tt bvp2.m} and increase the number of
tests done, e.g., by setting {\tt mvals = round(logspace(1,3,50));}
to do 50 tests for values of $m$ between 10 and 1000.
\end{enumerate}

