
\exercise[(trapezoid method for advection)]{10.4}

Consider the method
\eqlex{a}
U_j^{n+1} = U_j^n - \frac{ak}{2h}(U_j^n-U_{j-1}^n + U_j^{n+1}-U_{j-1}^{n+1}).
\end{equation}
for the advection equation $u_t+au_x=0$ on $0\leq x \leq 1$ with periodic
boundary conditions.  

\begin{enumerate}
\item This method can be viewed as the trapezoidal method applied to an ODE
system $U'(t) = AU(t)$ arising from a method of lines discretization of the
advection equation.  What is the matrix $A$?  Don't forget the boundary
conditions.

\item Suppose we want to fix the Courant number $ak/h$ as $k,~h\goto 0$.
For what range of Courant numbers will the method be stable if $a>0$?
If $a<0$?  Justify your answers in terms of eigenvalues of the matrix $A$
from part (a) and  the stability regions of the trapezoidal method.

\item Apply von Neumann stability analysis to the method \eqnex{a}.
What is the amplification factor $g(\xi)$?

\item For what range of $ak/h$ will the CFL condition be satisfied for this
method (with periodic boundary conditions)?

\item Suppose we use the same method \eqnex{a} for the initial-boundary value
problem with $u(0,t)=g_0(t)$ specified.  Since the method has a one-sided
stencil, no numerical boundary condition is needed at the right boundary
(the formula \eqnex{a} can be applied at $x_{m+1}$).  For what range of $ak/h$
will the CFL condition be satisfied in this case?  What are the eigenvalues
of the $A$ matrix for this case and when will the method be stable?

\end{enumerate} 


