%%
%% This is file `pst-ode.tex',
%%
%% Alexander Grahn, (C) 2012, 2013
%%
%% This program can be redistributed and/or modified under the terms
%% of the LaTeX Project Public License Distributed from CTAN archives
%% in directory macros/latex/base/lppl.txt.
%%
%% `pst-ode' defines \pstODEsolve for integrating systems of first order
%% ODEs using the Runge-Kutta-Fehlberg (RKF45) method with automatic
%% step size adjustment
%%
\def\fileversion{0.7}
\def\filedate{2014/03/27}

\csname PSTODELoaded\endcsname
\let\PSTODELoaded\endinput
%
% Requires some packages
\ifx\PSTricksLoaded\endinput\else \input pstricks \fi
%
\message{`pst-ode' v\fileversion, \filedate\space (ag)}
%
\edef\PstAtCode{\the\catcode`\@} \catcode`\@=11\relax
\pst@addfams{pst-ode}
%% prologue for postcript
\pstheader{pst-ode.pro}%

%%%%%%%%%%%%%%%%%%%%%%%%%%%%%%%%%%%%%%%%%%%%%%%%%%%%%%%%%%%%%%%%%%%%%%%%%%%%%%%%
%
% pstODEsolve
%
% LaTeX command for integrating systems of first order ODEs using the Runge-
% Kutta-Fehlberg method with automatic step size adjustment;
% values of the integration parameter `t' as well as the solution (= state)
% vectors `x(t)' at output points are stored as a long list in a Postscript
% object; its content can be plotted using the listplot* functions
% of pst-plot and pst-3dplot packages.
%
% Optionally, the result can be written to a file (ps2pdf -dNOSAFER ...)
%
% Usage:
%
% \pstODEsolve[Options]
%          {result}{output vector format}{ta}{tb}{number of output points}
%          {initial cond.}{function}
%
% options:
%   * append       result appended to <result> which must already exist (e. g.
%                  from previous use of \pstODEsolve); usually the initial
%                  condition vector argument is left empty in order to continue
%                  integration from the last state
%   * saveData     result is written to file <result>.dat
%   * algebraic    (system of) ODE given in infix notation
%   * algebraicIC  initial condition vector given in infix notation
%   * algebraicT   integration interval limits ta, tb given in infix notation
%   * algebraicOutputFormat output vector format given in infix notation
%   * algebraicAll output vector format, ta, tb, initial cond., function given
%                  in infix notation
%   * silent       suppress output of stepping information
%   * varsteptol   relative tolerance for step size adjustment
%
% arguments:
%
% #1: Postscript identifier taking the result as a long list of state vectors
%     calculated at the output points
% #2: output vector format, e. g. `(t) 0 1'; specifies which data to be written
%     to #1; (t) (parentheses required) puts value of integration parameter `t'
%     into the output list; 0, 1, 2, etc. specify the elements of the state
%     vector to be put into the output list
% #3: start value of integration parameter (ta)
% #4: end value of integration parameter (tb)
% #5: number of output points, including ta and tb; must be >= 2
% #6: initial condition vector; if empty, continue integration from the last
%     state vector of the previous \pstODEsolve usage or from state vector set
%     by \pstODEsetOrRestoreState macro
% #7: right hand side of ODE system; if given in Postscript notation (algebraic=
%     false), the function provided should pop (in reverse order!) the elements
%     of the current state vector from the operand stack and push the first
%     derivatives (right hand side of ODE system) back to it; inside the
%     function, the integration parameter can be accessed using `t'
%%%%%%%%%%%%%%%%%%%%%%%%%%%%%%%%%%%%%%%%%%%%%%%%%%%%%%%%%%%%%%%%%%%%%%%%%%%%%%%%
\define@boolkey[psset]{pst-ode}[PstODE@]{append}[true]{}%
\define@boolkey[psset]{pst-ode}[PstODE@]{saveData}[true]{}%
\define@boolkey[psset]{pst-ode}[PstODE@]{algebraicT}[true]{}%
\define@boolkey[psset]{pst-ode}[PstODE@]{algebraicIC}[true]{}%
\define@boolkey[psset]{pst-ode}[PstODE@]{algebraicOutputFormat}[true]{}%
\define@boolkey[psset]{pst-ode}[PstODE@]{algebraicAll}[true]{}%
\define@boolkey[psset]{pst-ode}[PstODE@]{silent}[true]{}%
\define@key[psset]{pst-ode}{varsteptol}{\def\ode@varsteptol{#1}}%
\def\pstODEsolve{\def\pst@par{}\pst@object{pstODEsolve}}%
\def\pstODEsolve@i#1#2#3#4#5#6#7{%
  \pst@killglue%
  \begingroup%
  \use@par%
  \def\filemode{w}%
  \ifPstODE@append\def\filemode{a}\fi%
  \edef\ode@ta{#3{}}%
  \edef\ode@tb{#4{}}%
  \edef\ode@init{#6{}}%
  \edef\ode@init{\expandafter\ode@zapspace\ode@init\@nil}%
  \edef\ode@arg{#7{}}%
  \ifPstODE@algebraicAll%
    \PstODE@algebraicTtrue%
    \PstODE@algebraicICtrue%
    \PstODE@algebraicOutputFormattrue%
    \Pst@algebraictrue%
  \fi%
  \ifPstODE@algebraicOutputFormat\edef\ode@fmt{#2{}}\fi%
  \pstVerb{
    tx@odeDict begin
    \ifPstODE@silent
      userdict /odeprint systemdict /pop get put
    \else
      userdict /odeprint {print flush} put
    \fi
    /ode@tol \ode@varsteptol\space def % rel. tolerance for step size adjustment
    %process arguments
    \ifPstODE@saveData /statefile (#1.dat) (\filemode) file def \fi
    \ifPstODE@algebraicT
      tx@Dict begin (\expandafter\ode@zapspace\ode@ta\@nil)
        AlgParser cvx exec end ode@dict /tStart exch def end
      tx@Dict begin (\expandafter\ode@zapspace\ode@tb\@nil)
        AlgParser cvx exec end ode@dict /tEnd   exch def end
    \else
      tx@Dict begin 1 dict begin #3 end end ode@dict /tStart exch def end
      tx@Dict begin 1 dict begin #4 end end ode@dict /tEnd   exch def end
    \fi
    ode@dict /dt tEnd tStart sub #5\space 1 sub div def end % output step size
    \ifx\@empty#6\@empty\else
      \ifPstODE@algebraicIC
        true setglobal globaldict /ode@laststate [
          tx@Dict begin (\ode@init) AlgParser cvx
          exec end
        ] put false setglobal
      \else
        true setglobal
        globaldict /ode@laststate [tx@Dict begin 1 dict begin #6 end end] put
        false setglobal
      \fi
    \fi%
    ode@dict /xlength ode@laststate length def end % number of equations
    ode@dict /xlength1 xlength 1 add def end % number of equations plus 1
    \ifPst@algebraic
      /ode@rpn tx@Dict begin (\expandafter\ode@zapspace\ode@arg\@nil) AlgParser end cvx bind def
      /ODESET {%system of ODEs
        /x exch def /y x def tx@Dict begin ode@rpn end ode@dict xlength end array astore
      } bind def
    \else
      /ODESET {
        aload pop tx@Dict begin ode@foo begin #7 end end ode@dict xlength end array astore
      } bind def
      %ensure local scope of user defined variables in #7, from BlueBook, p. 133
      /ODESET load 4 1 dict put
    \fi
    \ifPstODE@algebraicOutputFormat
      /ode@fmtrpn tx@Dict begin (\expandafter\ode@zapspace\ode@fmt\@nil) AlgParser end cvx bind def
      /formatoutput {%
        ode@laststate /x exch def /y x def /t ode@dict tcur end def
        tx@Dict begin ode@fmtrpn end
      } bind def
    \else /formatoutput {[#2] assembleresult} def \fi%
    %perform ode integration
    (\string\n pstODEsolve RKF45 method; '-' failed step, '+' successful step, %
    'o' output step : \string\n) odeprint
    ode@dict
      /tcur tStart def % current parameter t value
      /tout tStart dt add def % next output t
      /ddt dt def % initial integration step size
    end
    \ifPstODE@append\else
      [
        [formatoutput]
        \ifPstODE@saveData dup writeresult \fi
        aload pop
        true setglobal
      ]
      globaldict exch /#1 exch cvx put
      false setglobal
      (o) odeprint
    \fi
    #5\space 1 sub {
      ode@laststate ODEINT [ exch aload pop true setglobal ]
      globaldict exch /ode@laststate exch put false setglobal
      [
        #1 [formatoutput]
        \ifPstODE@saveData dup writeresult \fi
        aload pop
        true setglobal
      ]
      globaldict exch /#1 exch cvx put
      false setglobal
    } repeat
    \ifPstODE@saveData statefile closefile \fi
    end % tx@odeDict
  }%
  \endgroup%
  \ignorespaces%
}

\def\ode@zapspace#1#2\@nil{% helper for stripping spaces from argument
  \ifx#1 \relax\else#1\fi%
  \ifx\@empty#2\@empty\else\ode@zapspace#2\@nil\fi%
}

%macro for saving last state vector into PS object
\def\pstODEsaveState#1{% #1: identifier
  \pstVerb{
    true setglobal
    globaldict /#1 [ode@laststate cvx exec] cvx put
    false setglobal
  }%
}

%macro for restoring state vector from PS object or setting it from
%literal vector
\def\pstODEsetOrRestoreState#1{% #1: identifier or
  \pstVerb{                  %       space separated list of numbers
    true setglobal
    globaldict /ode@laststate [#1] put
    false setglobal
  }%
}

\psset[pst-ode]{append=false,saveData=false,algebraicIC=false,algebraicT=false,
  silent=false,varsteptol=1e-6,algebraicOutputFormat=false,algebraicAll=false
}
\psset{algebraic=false}
\catcode`\@=\PstAtCode\relax

%% END: pst-ode.tex
\endinput
